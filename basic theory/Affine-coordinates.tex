\subsection{Affine coordinates}

Pairings can be computed over elliptic curves represented in any coordinate system, but popular choices have been homogeneous projective and affine coordinates, depending on the ratio between inversion and multiplication.

\subsubsection{Homogeneous Projective coordinates} 
The choice of projective coordinates has proven especially advantageous at the 128-bit security level for single pairing
computation, due to the typically large inversion/multiplication ratio in this setting. The tangent line evaluated at
$ P = (x_P , y_P)$ can be computed with the following formula:

\begin{equation}
    g_{2\phi(T)}(P) = -2YZy_p + 3X^2x_pw + (3b^{'}Z^2 - Y^2)w^3
\end{equation}

\subsubsection{Affine coordinates} 

The choice of affine coordinates has proven more useful at higher security levels and embedding degrees, due to the action of the norm
map on simplifying the computation of inverses at higher extensions. The main advantages of affine coordinates are the simplicity of implementation and
format of the line functions, allowing faster accumulation inside the Miller loop if the additional sparsity is exploited.

If $ T = (x_1, y_1)$ is a point in $E^t(F_{p2})$, one can compute the point $ 2T := T + T $ with the following formula:

\begin{equation}
    (1 / y_p)g_{2\phi(T)}(P) = 1 + (-x_p/y_p)\lambda w + (1 / y_p)(\lambda x_1 - y_1)w^3
\end{equation}

you can read this paper \cite{website:The-realm-of-the-pairings} to know more deatils about this, we big thanks to all contributors of 
PR \cite{website:PR} in BitVM2 \cite{website:BitVM2}.
