\subsection{Affine coordinates}

Pairings can be computed over elliptic curves represented in any coordinate system, but popular choices have been homogeneous projective and affine coordinates, depending on the ratio between inversion and multiplication.

\subsubsection{Homogeneous projective coordinates} 
The choice of projective coordinates has proven particularly advantageous at the 128-bit security level for single pairing
computation, due to the typically high inversion/multiplication ratio in this setting. The tangent line evaluated at
$ P = (x_P , y_P)$ can be computed with the following formula:

\begin{equation}
    g_{2\phi(T)}(P) = -2YZy_p + 3X^2x_pw + (3b^{'}Z^2 - Y^2)w^3
\end{equation}

\subsubsection{Affine coordinates} 

The choice of affine coordinates has proven more useful at higher security levels and embedding degrees, primarily due to 
the norm map's role in simplifying the computation of inverses at higher extensions. The main advantages of affine coordinates 
lie in their ease of implementation and the straightforward format of the line functions. These features enable faster 
accumulation within the Miller loop, particularly when additional sparsity is exploited.

If $ T = (x_1, y_1)$ is a point in $E^t(F_{p2})$, one can compute the point $ 2T := T + T $ with the following formula:

\begin{equation}
    (1 / y_p)g_{2\phi(T)}(P) = 1 + (-x_p/y_p)\lambda w + (1 / y_p)(\lambda x_1 - y_1)w^3
\end{equation}

You can read this paper \cite{website:The-realm-of-the-pairings} for more details on this topic. 
We extend our heartfelt thanks to all contributors of the PR \cite{website:PR} in BitVM2 \cite{website:BitVM2}.
