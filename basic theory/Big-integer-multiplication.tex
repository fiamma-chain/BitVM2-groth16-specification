\subsection{Big integer multiplication}

\subsubsection{Montgomery reduction}

Montgomery reduction \cite{website:Montgomery}, also known as REDC, is an algorithm that simultaneously computes the product by $R'$ and 
reduces modulo $N$ more quickly than the naïve method. Unlike conventional modular reduction 
which focuses on making the number smaller than $N$, Montgomery reduction aims at making the number more divisible by $R$. 
It achieves this by adding a sophisticatedly chosen small multiple of $N$ to cancel the residue modulo $R$. 
Dividing the result by $R$ yields a much smaller number. 
This number is so much smaller that it is close enough to the reduction modulo $N$, and 
computing the reduction modulo $N$ requires only a final conditional subtraction. 
Because all computations are done using only reduction and divisions with respect to $R$, not $N$, the algorithm runs faster than
straightforward modular reduction by division.


\subsubsection{Karatsuba multiplication}
Karatsuba multiplication \cite{website:Karatsuba} is a divide-and-conquer algorithm for (non-modular) multiplication that reduces the computational cost for n-bit integers from $O(n^2)$
in classical multiplication to $O(n^{\log_2^3})$ which is $O(n^{1.58…})$.

When applied to modular exponentiation with n-bit numbers, including the exponent, the cost is reduced from $O(n^3)$ to $O(n^{1+log_2^3})$ 
which is $O(n^{2.58…})$. One of several methods to leverage the benefits of Karatsuba multiplication during modular reduction is to pre-compute the (non-modular) 
inverse of the modulus to slightly more than n bits, achievable at a cost of $O(n^2)$, this is considered negligible in the context of $O$,
using classical algorithms.

Karatsuba multiplication can be used in conjunction with Montgomery reduction. We highly appreciate the work \cite{website:PR75} of Robin and the Zerosync team, which has reduced 
for multiplication to approximately 55\% of the native version.
