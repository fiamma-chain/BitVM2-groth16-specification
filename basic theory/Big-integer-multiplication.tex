\subsection{Big integer multiplication}

\subsubsection{Montgomery reduction}

Montgomery reduction \cite{website:Montgomery}, also known as REDC, is an algorithm that simultaneously computes the product by $R'$ and 
reduces modulo $N$ more quickly than the naïve method. Unlike conventional modular reduction, 
which focuses on making the number smaller than $N$, Montgomery reduction focuses on making the number more divisible by $R$. 
It does this by adding a small multiple of $N$ which is sophisticatedly chosen to cancel the residue modulo $R$. 
Dividing the result by $R$ yields a much smaller number. 
This number is so much smaller that it is nearly the reduction modulo $N$, and 
computing the reduction modulo $N$ requires only a final conditional subtraction. 
Because all computations are done using only reduction and divisions with respect to $R$, not $N$, the algorithm runs faster than
a straightforward modular reduction by division.


\subsubsection{Karatsuba multiplication}
Karatsuba multiplication \cite{website:Karatsuba} is a divide-and-conquer algorithm for (non-modular) multiplication, which for n-bit integers reduces cost from the $O(n^2)$
for classical multiplication to $O(n^{\log_2^3})$, that is $O(n^{1.58…})$.

Applied to modular exponentiation with n-bit numbers including exponent, the cost goes from $O(n^3)$ to $O(n^{1+log_2^3})$
, that is $O(n^{2.58…})$. One of several methods for getting the benefits of Karatsuba multiplication during modular reduction is pre-computing the (non-modular) 
inverse of the modulus to slightly more than n bits, which can be done at cost $O(n^2)$ (thus negligible as far as $O$
 is concerned) with classical algorithms.

Karatsuba multiplication can be used together with Montgomery reduction. We highly appreciate at the work \cite{website:PR75} of Robin and Zerosync team. It reduce
the script size of multiplication to around 55\% of native version.
