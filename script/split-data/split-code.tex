\subsubsection{Split code}

\lstdefinestyle{mystyle}{
    keywordstyle= \color{ blue!70},			
    commentstyle= \color{red!50!green!50!blue!50},		
    numberstyle=\tiny\color{codegray},		
    stringstyle=\color{codepurple},
    basicstyle=\ttfamily\footnotesize,
    breakatwhitespace=false,         
    breaklines=true,	  
    captionpos=b,                    
    keepspaces=true,                 
    numbers=left,		            
    numbersep=5pt,                  
    showspaces=false,                
    showstringspaces=false,		
    showtabs=false,                  
    tabsize=2, 
    frame=shadowbox,	
}

We will give an example to explain how the split process works based on $ Fn \ \ double\_ell\_affine() $, there have 2 types of code:
\begin{itemize}
    \item Split\_code: it will generate all subscripts;
    \item Witness\_code: it will generate all inputs and outputs for all subscripts
\end{itemize}

\paragraph*{Split\_pseudocode}

\begin{lstlisting}

    pub fn split_double_ell_affine() -> Vec<Script> {
        let mut res = vec![];

        // [p.c0, p.c1, c3, c4, lamda, mu, Q.x, Q.y, 1]
        // [p.c0, p.c1, c3, c4, x3, y3]
        // [final_c0] | [y3, x3]
        // [final_c0, x3, y3]
        res.push(script! {

            { Pairing::double_line_g2_optimization() }

            { Fq2::toaltstack() }
            { Fq2::toaltstack() }

            // [p.c0, p.c1, c3, c4,]
            // compute b = p.c1 * (c3, c4)
            { Fq6::mul_by_01() }
            // [p, c3, c4, b]

            // [p.c0, b]
            { Fq12::mul_fq6_by_nonresidue() }
            // [p.c0, b * beta]

            // compute final c0 = a + beta * b
            { Fq6::add(6, 0) }

            { Fq2::fromaltstack() }
            { Fq2::fromaltstack() }

        });

        // [c0, c3, c4, b, a, p.c0, p.c1, lamda, p.x, -mu, p.y]
        // [c0, c3, c4, b, a, p.c0, p.c1, p.y, -mu, lamda * p.x]
        // [c0, c3, c4, b, a, p.c0, p.c1, lamda * p.x, p.y * -mu]
        // [c0, c3, c4, b, a, p.c0 + p.c1]
        // [c3, c4, b, a, e, c0]
        // [c4, b, a, e, c0, c3]
        // [c4, b, a, e, c0 + c3]
        // [b, a, e, c0 + c3, c4]
        // [b, a, e]
        // [e, a + b]
        // [final_c1]
        res.push(script! {

            { Fq2::mul_by_fq(1, 0) }
            { Fq2::copy(29) }
            { Fq2::equalverify() }
            { Fq2::mul_by_fq(1, 0) }
            { Fq2::copy(28) }
            { Fq2::equalverify() }

            { Fq6::add(6, 0) }
            { Fq2::roll(22) }
            { Fq2::roll(22) }
            { Fq2::add(2, 0) }
            { Fq2::roll(20) }
            { Fq6::mul_by_01() }
            { Fq6::add(12, 6) }
            { Fq6::sub(6, 0) }
        });
        unsafe {
            FQ12_SPLIT_DOUBLE_ELL_COUNTER += 1;
        }
        res
    }

\end{lstlisting}

\paragraph*{Witness\_code}

\begin{lstlisting}

    pub fn witness_add_ell_affine(
        f: &Fq12,
        p1: &ark_G1Affine,
        affi_t4: &G2Affine,
        affi_q4: &G2Affine,
    ) -> (Fq12, Vec<ScriptContext>, G2Affine) {
        let mut contexts = vec![];
        //////////////////// script-1 ///////////////
        // non-fixed (non-constant part) P4
        let (lambda, mu, x3, y3) = Self::line_add_g2(affi_t4, affi_q4);

        let affi_t4_new = G2Affine::new(x3, y3);

        let mut p = ark_G1Affine::new(p1.x, p1.y);

        p.x = -p1.x / p1.y;
        p.y = Fq::from(1) / p1.y;
        //////////////////// script-1 ///////////////
        let mut context = ScriptContext::default();

        let c0 = ark_bn254::Fq2::from(1);

        let mut c3 = lambda.clone();
        c3.mul_assign_by_fp(&p.x);

        let mut c4 = -mu.clone();
        c4.mul_assign_by_fp(&p.y);

        let mut b = f.c1 as Fq6;
        b.mul_by_01(&c3, &c4);

        let mut a = f.c0 as Fq6;
        a.mul_by_fp2(&c0);

        let mut beta_b = b;
        <Fq12Config as Fp12Config>::mul_fp6_by_nonresidue_in_place(&mut beta_b);

        let final_c0 = a + beta_b;

        // fill inputs for subscript - 1
        context
            .inputs
            .extend(f.c0.to_base_prime_field_elements().collect::<Vec<Fq>>());
        ...
        // fill outputs for subscript - 1
        context
            .outputs
            .extend(final_c0.to_base_prime_field_elements().collect::<Vec<Fq>>());
        ...

        //////////////////// script-2 ///////////////
        let mut context = ScriptContext::default();
        // fill inputs for subscript - 2
        context
            .inputs
            .extend(c0.to_base_prime_field_elements().collect::<Vec<Fq>>());
        ...

        let e = f.c0 + f.c1;
        let mut e = e;
        let c0_c1 = c0 + c3;
        e.mul_by_01(&c0_c1, &c4);

        let a_b = a + b;
        let final_c1 = e - a_b;

        // fill outputs for subscript - 2
        context
            .outputs
            .extend(final_c1.to_base_prime_field_elements().collect::<Vec<Fq>>());
        contexts.push(context);

        (Fq12::new(final_c0, final_c1), contexts, affi_t4_new)
    }

\end{lstlisting}
