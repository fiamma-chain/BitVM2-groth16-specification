\subsection{Split data} \label{sec:split-data}

We present the results directly on how we split the script whose size exceeds the 4M limitation, as demonstrated in \ref{sec:benchmark data},
there are only some operations of $F_{q12}$ that need to be split after we optimize the operations for $G_1$ and $G_2$

\begin{center}
\begin{tabular}{|c|c|c|c|} \hline
    operator type & script size & max depth & exceed 4M? \\ \hline
    $F_{q12}: a * b$ & \textcolor{red}{11,641,775} bytes & 545 & yes \\ \hline
    $F_{q12}: frobenius\_map(1)$ & \textcolor{red}{4,541,887} bytes & - & yes \\ \hline
    $F_{q12}: mul\_by\_034$ & \textcolor{red}{9,810,459} bytes & - & yes \\ \hline
    $F_{q12}: ell\_by\_constant$ & \textcolor{red}{9,525,050} bytes & 383 & yes \\ \hline    
\end{tabular}
\end{center}


We will demonstrate how we manually split these four large scripts one by one, striving to satisfy the following properties concurrently:

\begin{itemize}
    \item Ensuring that size and stack depth limitations are not exceeded;
    \item Minimizing the size of inputs and outputs;
    \item Making the logic of each chunk as clear and readable as possible; 
\end{itemize}

\subsection{split-code}

\lstdefinestyle{mystyle}{
    keywordstyle= \color{ blue!70},			
    commentstyle= \color{red!50!green!50!blue!50},		
    numberstyle=\tiny\color{codegray},		
    stringstyle=\color{codepurple},
    basicstyle=\ttfamily\footnotesize,
    breakatwhitespace=false,         
    breaklines=true,	  
    captionpos=b,                    
    keepspaces=true,                 
    numbers=left,		            
    numbersep=5pt,                  
    showspaces=false,                
    showstringspaces=false,		
    showtabs=false,                  
    tabsize=2, 
    frame=shadowbox,	
}

\subsubsection{$F_{q12} : a \cdot b$}

\begin{lstlisting}

    % Split Fq12 mul into small scripts. For each script
    % size < 4M && max_stack_used < 1000
    % Input: a0, a1, b0, b1
    %
    % Algorithm:
    %     Final_a0 = a0 * b0 + a1 * b1 * \gamma
    %     Final_a1 = (a0 + a1) * (b0 + b1) - (a0 * b0 + a1 * b1)
    pub fn split_mul() -> Vec<Script> {
        % The degree-12 extension on BN254 Fq6 is under the polynomial z^2 - y

        let mut res = vec![];

        res.push(script! {
            % a0, b0
            { Fq6::mul(6, 0) }
            % a0 * b0
        });

        res.push(script! {
            % a1, b1
            { Fq6::mul(6, 0) }
            % a1 * b1
        });

        res.push(script! {
            % a0 * b0, a1 * b1, a0, a1, b0, b1,
            { Fq6::add(6, 0) }
            % a0 * b0, a1 * b1, a0, a1, b0 + b1,
            { Fq6::add(12, 6) }
            % a0 * b0, a1 * b1, b0 + b1, a0 + a1,
            { Fq6::mul(6, 0) }
            % a0 * b0, a1 * b1, (a0 + a1) * (b0 + b1)
            { Fq6::copy(12) }
            % a0 * b0, a1 * b1, (a0 + a1) * (b0 + b1), a0 * b0
            { Fq6::copy(12) }
            % a0 * b0, a1 * b1, (a0 + a1) * (b0 + b1), a0 * b0, a1 * b1
            { Fq12::mul_fq6_by_nonresidue() }
            % a0 * b0, a1 * b1, (a0 + a1) * (b0 + b1), a0 * b0, a1 * b1 * \gamma
            % z^2 - \gamma = 0
            { Fq6::add(6, 0) }
            % a0 * b0, a1 * b1, (a0 + a1) * (b0 + b1), a0 * b0 + a1 * b1 * \gamma
            { Fq6::add(18, 12)}
            % (a0 + a1) * (b0 + b1), a0 * b0 + a1 * b1 * \gamma, a0 * b0 + a1 * b1
            { Fq6::sub(12, 0) }
            % a0 * b0 + a1 * b1 * \gamma, (a0 + a1) * (b0 + b1) - (a0 * b0 + a1 * b1)
        });

        res
    }

\end{lstlisting}

\subsubsection{$F_{q12} : frobenius\_map(1)$}

\begin{lstlisting}

    pub fn split_frobenius_map(i: usize) -> Vec<Script> {
        let mut res = vec![];
        if i == 1 {
            % [p.c0, p.c1]
            res.push(script! {
                { Fq6::frobenius_map(i) }
                { Fq6::roll(6) }
                { Fq6::frobenius_map(i) }
                % [p.c1 ^ p^i, p.c0 ^ p^i]
            });
            % [p.c1 ^ p^i]
            res.push(Fq6::mul_by_fp2_constant(
                &ark_bn254::Fq12Config::FROBENIUS_COEFF_FP12_C1
                    [i % ark_bn254::Fq12Config::FROBENIUS_COEFF_FP12_C1.len()],
            ));
        } else {
            res.push(Self::frobenius_map(i));
        }

        res
    }
\end{lstlisting}

\subsubsection{$F_{q12}: mul\_by\_034$}

\begin{lstlisting}

    pub fn split_mul_by_034() -> Vec<Script> {
        let mut res = vec![];

        % compute b = p.c1 * (c3, c4)
        % [p.c1, c3, c4]
        res.push(Fq6::mul_by_01());
        % [b]

        % [c0, c3, b, p.c0, p.c1]
        % [Fq2, Fq2, Fq6, Fq6, Fq6]
        res.push(script! {
            % compute a = c0 * p.c0
            { Fq6::copy(6) }
            % [c0, c3, b, p.c0, p.c1, p.c0]
            { Fq2::copy(26) }
            % [c0, c3, b, p.c0, p.c1, p.c0, c0]
            { Fq6::mul_by_fp2() }
            % [c0, c3, b, p.c0, p.c1, c0 * p.c0]
            % [c0, c3, b, p.c0, p.c1, a]
            % compute gamma * b
            { Fq6::roll(18) }
            % [c0, c3, p.c0, p.c1, a, b]
            { Fq12::mul_fq6_by_nonresidue() }
            % [c0, c3, p.c0, p.c1, a, b * gamma]

            % compute final c0 = a + gamma * b
            % [c0, c3, p.c0, p.c1, a, b * gamma]
            { Fq6::copy(6) }
            % [c0, c3, p.c0, p.c1, a, b * gamma, a]
            { Fq6::add(6, 0) }
            % [c0, c3, p.c0, p.c1, a, a + b * gamma]
            % [c0, c3, p.c0, p.c1, a, final_c0]

            % compute e = p.c0 + p.c1
            { Fq6::add(18, 12) }
            % [c0, c3, a, final_c0, p.c0 + p.c1]
            % [c0, c3, a, final_c0, e]

            % compute c0 + c3
            { Fq2::add(20, 18) }
            % [a, final_c0, e, c0 + c3]
        });

        % [b, a, final_c0, e, c0 + c3, c4]
        res.push(script! {
            % update e = e * (c0 + c3, c4)
            { Fq6::mul_by_01() }
            % [b, a, final_c0, e]

            % sum a and b
            { Fq6::add(18, 12) }
            % [final_c0, e, b + a]

            % compute final c1 = e - (a + b)
            { Fq6::sub(6, 0) }
            % [final_c0, e - (b + a)]
            % [final_c0, final_c1]
        });

        res
    }
    
\end{lstlisting}


\subsubsection{$F_{q12}: mul\_by\_constant$}

\begin{lstlisting}

    pub fn split_mul_by_034_with_4_constant(constant: &ark_bn254::Fq2) -> Vec<Script> {
        let mut res = vec![];

        % [p.c1, c3], constant = c4
        res.push(Fq6::mul_by_01_with_1_constant(constant));

        % compute a = p.c0 * c0
        % Input: [p.c0, c0]
        % Output: [p.c0 * c0]
        res.push(Fq6::mul_by_fp2());

        % [c0, c3, p.c0, p.c1, a, b]
        res.push(script! {
            { Fq6::copy(0) }
            % [c0, c3, p.c0, p.c1, a, b, b]
            % compute beta * b
            { Fq12::mul_fq6_by_nonresidue() }
            % [c0, c3, p.c0, p.c1, a, b, b * beta]

            % compute final c0 = a + beta * b
            { Fq6::copy(12) }
            % [c0, c3, p.c0, p.c1, a, b, b * beta, a]
            { Fq6::add(6, 0) }
            % [c0, c3, p.c0, p.c1, a, b, a + beta * b]
            % [c0, c3, p.c0, p.c1, a, b, final_c0]

            % compute e = p.c0 + p.c1
            { Fq6::add(24, 18) }
            % [c0, c3, a, b, final_c0, e]

            % compute c0 + c3
            { Fq2::add(26, 24) }
            % [a, b, final_c0, e, c0 + c3]

            % update e = e * (c0 + c3, c4)
            { Fq6::mul_by_01_with_1_constant(constant) }
            % [a, b, final_c0, e]

            % sum a and b
            { Fq6::add(18, 12) }
            % [final_c0, e, a + b]

            % compute final c1 = e - (a + b)
            { Fq6::sub(6, 0) }
            % [final_c0, final_c1]
        });

        res
    }

\end{lstlisting}
\subsubsection{split result}

We present the split result directly in the following table:

\begin{center}
\begin{tabular}{|c|c|c|c|c|} \hline
    operator typr & subscript & script size & max depth & exceed 4M? \\ \hline
    $F_{q12}: a * b$ & & \textcolor{red}{6,727,971} bytes & 545 & yes \\ \hline
     & chunk0 & \textcolor{green}{ 3,085,202 } bytes & 545 & no \\ \hline
     & chunk1 & \textcolor{green}{ 3,642,594} bytes & 545 & no \\ \hline
    $F_{q12}: a * a$ & & \textcolor{red}{4,495,790} bytes & - & yes \\ \hline
    & chunk0 & \textcolor{green}{2,246,763} bytes & 545 & no \\ \hline
    & chunk1 & \textcolor{green}{2,249,027} bytes & 545 & no \\ \hline
    $F_{q12}: doube_ell $ & & \textcolor{red}{6,416,951} bytes & - & yes \\ \hline
    & chunk0 & \textcolor{green}{3,708,344} bytes & 545 & no \\ \hline
    & chunk1 & \textcolor{green}{2,708,607} bytes & 545 & no \\ \hline
    $F_{q12}: add_ell $ & & \textcolor{red}{6,415,463} bytes & - & yes \\ \hline
    & chunk0 & \textcolor{green}{3,706,856} bytes & 545 & no \\ \hline
    & chunk1 & \textcolor{green}{2,708,607} bytes & 545 & no \\ \hline
    $F_{q12}: ell\_by\_constant$ & & \textcolor{red}{4,714,973} bytes & 383 & yes \\ \hline
    & chunk0 & \textcolor{green}{2,568,521} bytes & 545 & no \\ \hline
    & chunk1 & \textcolor{green}{2,145,297} bytes & 545 & no \\ \hline
\end{tabular}
\end{center}

\textbf{Note: we combine Fq12\_ell and double\_g2, add\_g2 operation to get a smaller subscript number}

