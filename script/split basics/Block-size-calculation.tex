\subsubsection{Block-size-calculation}
Understanding the calculation of Bitcoin block size following the Taproot upgrade is essential.

\paragraph{block size} 

The following illustration depicts the block size calculation:

\begin{figure}[ht] 
    \centering  
    \includegraphics[width=0.85\columnwidth]{images/block-size.png} 
    \caption{Block size}
    \label{fig:block-size}
\end{figure}

\paragraph*{transaction size} 

The transaction size calculation is illustrated below:

\begin{figure}[ht] 
    \centering  
    \includegraphics[width=0.85\columnwidth]{images/transaction-size.png} 
    \caption{Transaction size}
    \label{fig:transaction-size}
\end{figure}

For more detailed information, please refer to \cite{website:transaction-size}

\paragraph*{script chunk limitation} 

Based on the design of BitVM2 \cite{website:BitVM2}, We aim for each script chunk to be packed into one block as a transaction.
So, the transaction size could not exceed 4,000,000 - 320 = 3,999,680 weight units \cite{website:transaction-size}.

A disputed transaction, characterized by 1 input and 2 outputs, typically has an average non-witness data size of approximately 464 weight units. 
Consequently, the witness size limitation for such transactions is calculated as 3,999,680 - 464 = 3,999,216 weight units.

As outlined in BitVM2 \cite{website:BitVM2}, the disputed transaction needs the signature of Committee, and the signature type is
SIGNHASH\_SINGLE. Let's assume that the number of Committee is 7 and the size of each schnorr signature is 65 bytes (64 bytes for SIGNHASH\_ALL)

So the limitation will be 3,999,216 - 7 * 65 - 8(stack item size) = 3,998,753 weight units.

\begin{figure}[ht] 
    \centering  
    \includegraphics[width=0.85\columnwidth]{images/ZKP-script-chunk-limitation.png} 
    \caption{ZKP script chunk limitation}
    \label{fig:ZKP-script-chunk-limitation}
\end{figure}

As for the structure of subscript, the normal flow should be as follows:

\begin{itemize}
    \item check the hash \cite{website:blake3} is consistent with the input and output
    \item check the Winternitz \cite{website:witernitz} signature
    \item execute the subscript
\end{itemize}

So, if we take these factors into account, the subscript structure should be like the following picture

\begin{figure}[ht] 
    \centering  
    \includegraphics[width=0.85\columnwidth]{images/structure-of-subscript.png} 
    \caption{structure of subscript}
    \label{fig:structure of subscript}
\end{figure}

\textbf{It should be noted that: we need to reduce the maxmial input and putput size of subscript less than 1024 to invoke at most 2 rounds blake3 \cite{website:blake3} hash progress}
